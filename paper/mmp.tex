For each language predicate $p$, a classifier $D_p$ was learned to decide
whether objects possessed the attribute denoted by $p$.  This classifier was
informed by sub-classifiers that determined whether $p$ held for a particular
subset of the features used to describe objects.

The feature space of objects was partitioned according to the robot behavior,
$b\in B$, used to gather the data for those features, as well as the modality,
$m\in M$, to which that data belonged.  Prior work has shown this approach to
be an effective method for multi-modal learning in robots
~\cite{sinapov:icra14}.  A behavior together with a modality describe a
\textit{context}.  Table~\ref{tab:feature_space_of_contexts} gives the
behaviors, modalities, and the number of features associated with each context.
Each context classifier $C_{bm}$ is a quadratic-kernel SVM trained with the
positively and negatively labelled context feature vectors derived from the
\ispy game (section~\ref{ssec:gll}).  Given an object $o\in O$ from the set of
objects, the features relevant for context classifier $C_{bm}$ are denoted
$o_{bm}$.  For each object, there were a number of \textit{observations} for
each behavior.  For example, there were 6 angles of images used to compute
features in the \textbf{look} behavior, and the robot repeated other behaviors
like \textbf{drop} 6 times to gather sufficient data.  Finally,
$C_{bm}(o_{bm})\in [-1,1]$ is the average classifier output over all
observations for $o$, where individual SVM decisions on observations were in
$\{-1,1\}$. Then, for each context, $bm$, we calculated the Cohen's Kappa
$\kappa_{bm}$ to measure the agreement across observations between the
decisions of the $C_{bm}$ classifier and the ground truth labels from the \ispy
game, in order to determine a confidence in $[0,1]$ for each context.  We
ceiling negative $\kappa$ to $0$.

Given these context classifiers and associated confidences, we calculate an
overall decision confidence, $D_p(o)$,
for $o\in O$ for each behavior $b$ and modality $m$ as:
\begin{equation}
	D_p(o) = \sum_{b\in B,m\in M}{\kappa_{bm} C_{bm}(o_{bm})} \in [-1,1]
\end{equation}
The sign of $D_p(o)$ gives a decision on whether $p$ applies to $o$ with
confidence $|D_p(o)|$.
