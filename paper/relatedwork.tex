% relation to grounded language learning tasks
Researchers have made substantial progress on grounding language for robots, enabling tasks such as object recognition and route following from verbal descriptions.
Early work used vision together with speech descriptions of objects to learn grounded semantics~\cite{roy:cogsci02}.

In the past few years, much of this work has focused on combining language with visual information.
For grounding referring expressions in an environment, many learn perceptual classifiers for words such as `red', `square', and `left' given some pairing of human descriptions and labeled scenes~\cite{liu:acl14,malinowski:nips14,mohan:acs13,sun:icra13,dindo:iros10,vogel:aaai10}.
Some approaches additionally incorporate language models into the learning phase~\cite{spranger:ijcai15,krishnamurthy:acl13,perera:aaai13,matuszek:icml12}.
Incorporating a language model also allows for more robust generation of robot referring expressions for objects, as explored in~\cite{tellex:rss14}. Our method uses simple language understanding and constructs new predicate classifiers for each unseen context word used by a human playing \ispy, and our generation system for describing objects is based only on these predicate classifiers.

Outside of robotics, there has been some work on combining language with sensory modalities other than vision, such as audio~\cite{kiela:emnlp15}.
Unlike that line of work, our system is embodied in a learning robot that manipulates objects to gain non-visual sensory experience.

Including a human in the learning loop provides a more realistic learning scenario for applications such as household and office robotics.
Past work has used human speech plus gestures describing sets of objects on a table as supervision to learn attribute classifiers~\cite{matuszek:aaai14,kollar:rss13}. Recent work introduced the \ispy game as a supervisory framework for grounded language learning~\cite{parde:ijcai15}.
Our work differs from these prior projects in that we use additional sensory data beyond vision to build object attribute classifiers. Additionally, in our instantiation of the \ispy task, the robot and the human both take a turn describing objects, where in previous work~\cite{parde:ijcai15} only humans gave descriptions.
