In our \ispy task, the human and robot take turns describing objects from among four on a tabletop [[Figure picture??]].
There are two rounds of turns in each game. In each round, the human subject starts by describing and object of her choice.

Subjects were asked to describe objects through their attributes, as opposed to singling them out as instances.
As an example, we suggested subjects describe an object as ``black rectangle'' as opposed to ``whiteboard eraser.''
Additionally, subjects were told they could handle the objects physically before offering a description, but were not asked to use non-visual predicates.
Once subjects offered a description, the robot guessed the objects it most thought the subject was talking about in order (see section~\ref{ssec:gll}) until one was confirmed correct.

In the second half of each round, the robot picked an object and then described it with up to three predicates (see section~\ref{ssec:gll}).
The subject was again able to pick up and physically handle objects before guessing.
The robot confirmed or denied each subject guess  until the right object was chosen.

The \ispy game admits two clear metrics.
The \textit{robot guess} metric is the number of turns the robot took to guess what object the subject was describing.
The \textit{human guess} metric is the number of turns the human took to guess what object the robot was describing.
Using these metrics, we compare the performance of two \ispy playing systems as described in section ~\ref{sec:experiment}.