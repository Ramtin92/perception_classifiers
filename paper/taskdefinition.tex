In our \ispy task, the human and robot take turns describing objects from among
four on a tabletop [[Figure picture??]].  Subjects were asked to describe
objects using attributes rather than category labels.  As an example, we
suggested subjects describe an object as ``black rectangle'' as opposed to
``whiteboard eraser.''  Additionally, subjects were told they could handle the
objects physically before offering a description, but were not explicitly asked
to use non-visual predicates.  Once subjects offered a description, the robot
guessed the object it believed the subject was describing, generating
candidates in order of their computed confidence (see section~\ref{ssec:gll})
until one was confirmed correct.

In the second half of each round, the robot picked an object and then described
it with up to three predicates (see section~\ref{ssec:gll}).  The subject was
again able to pick up and physically handle objects before guessing.  The robot
confirmed or denied each subject guess until the right object was chosen.

The \ispy game admits two clear metrics.  The \textbf{robot guess} metric is
the number of turns the robot took to guess what object the subject was
describing.  The \textbf{human guess} metric is the number of turns the human
took to guess what object the robot was describing.  Using these metrics, we
compare the performance of two \ispy playing systems (multi-modal
vs. vision-only) as described in section ~\ref{sec:experiment}.
