\paragraph{Data Folds.}
We divided our 32-object dataset into 8 folds.  For each fold, 10 human
subjects played \ispy with both the \textbf{vision only} and
\textbf{multi-modal} systems.  Four games were played by each subject.  The
\textbf{vision only} system and \textbf{multi-modal} system were each used in
two games (once as a the describer and once as the guesser), and their temporal
order was randomized.  Each system played with all 8 objects, but the split and
the order of objects on the table were randomized.

For fold 0, the systems were undifferentiated and so only one set of two games
was played by each subject.  For subsequent folds, the systems were
incrementally trained using labels from previous folds only, such that the
systems were always being tested against novel, unseen objects.  In the prior
work using the \ispy game~\cite{parde:ijcai15}, the same objects were used
during training and test, and therefore it did not test the ability of the
learned predicate classifiers to actually generalize to novel objects.

\paragraph{Human Subjects.}

Our 40 subjects were undergraduate and graduate students as well as some staff
at our university.  We had XX male and XX female study participants aged
XX[freshmen] to XX[chempostdocs].

At the beginning of each trial, subjects were shown an instructional video of
one of the authors playing a single game of \ispy with the robot, then given a
sheet of instructions about the task's turn-taking and how to communicate with
the robot.  In every game, subjects took one turn and the robot took one turn.

A study coordinator remained in the room to deal with system mechanical and
software failures, both of which resulted in the current game starting over.
To avoid noise from automatic speech recognition, the study coordinator also
transcribed the subject's speech to the system from a remote computer, but this
was done discretely and not revealed to the subject until debriefing when the
games were over.
