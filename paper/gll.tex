Language predicates and positive/negative labels for those predicates were gathered through human-robot dialog during the \ispy game.
The human subject and robot were seated at opposite ends of a small table.
A set of 4 objects were placed on the table for both to see.
We denote the objects on the table during a given game as set $\mathcal{O}_T$.

\paragraph{Human Turn.} On the subject's turn, the robot asked him or her to pick an object and describe it in one phrase.
We used a standard stopword list to strip out non-content words from the subject's description.
The remaining words were treated as a set of language predicates, $\mathcal{H}_p$.
The robot assigned scores $S$ to each object $i\in \mathcal{O}_T$ on the table.
\begin{equation}
	S(i) = \sum_{p\in \mathcal{H}_p}{G_p(i)}
\end{equation}
The robot guessed objects in descending order by score (ties broken randomly) by pointing at them and asking whether it was correct.
When the correct object was found, it was added as a positive training example for all classifiers  $p\in \mathcal{H}_p$ for use in future training.

\paragraph{Robot Turn.} On the robot's turn, an object was chosen at random from those on the table.
To describe the object, the robot scored the set of known predicates learned from previous play.
Following Gricean principles~\cite{grice:bkchapter75}, the robot attempted to describe the object with predicates that applied to it but did not ambiguously refer to other objects on the table.
We use a predicate score $R$ that rewards describing the chosen object $i^*$ and penalizes describing the other objects on the table.
\begin{equation}
	R(p) = |O_T|G_p(i^*) - \sum_{j\in{\mathcal{O}_T}\setminus\{i^*\}}{G_p(j)}
\end{equation}
Note that this scoring equation also rewards predicates for describing $i^*$ while clearly {\it not} describing the other objects on the table (\textit{e.g.}  $G_p(j)<0$).
The robot choose up to three highest scoring predicates $\hat{P}$ to describe object $i^*$ to the subject, using fewer if $S<0$ for the remaining predicates.
Once they were ready to guess, the subject touched objects until the robot confirmed that they had guessed the right one ($i^*$).

The robot then pointed to $i^*$ and engaged the user in a brief follow-up dialog in order to gather both positive and negative labels for $i^*$.
In addition to predicates $\hat{P}$ used to describe the object, the robot selected two additional predicates $\bar{P}$.
$\bar{P}$ were selected randomly with $p\in P$ having a chance of inclusion proportional to $1-|G_p(i^*)|$, such that classifiers with low confidence in whether or not $p$ applied to $i^*$ were more likely to be selected.
The robot then asked the subject whether they would describe the object $i^*$ using each $p\in\hat{P}\cup\bar{P}$.
The subject's responses to these questions provided additional positive/negative labels to the classifiers of these predicates for use in future training.
