\documentclass{article} 
\usepackage{fullpage}

\begin{document}

\section{Initial Setup}

\paragraph{} Launch the \texttt{segbot\_arm} according to standard instructions. Then launch the ispy robot nodes with \texttt{roslaunch perception\_classifiers robot\_ispy.launch}

\section{Setup Per User}

\paragraph{} Go to \texttt{http://www.cs.utexas.edu/users/jesse/ispy/new\_session.php} and enter the user's name and the current fold in the blanks at the top of the table, then click the ``Add'' button. If the user participated in an earlier fold, try to use the name from before. You will now see the user's ID, name, and the object lists with pictures for reference for the first and second rounds for this fold. Note that the user ID changes between fold rounds.

\paragraph{} Find the objects with the 4 IDs given for fold ``a'' and arrange them from left-to-right on the table from the robot's perspective in the same order in which they are given.

\paragraph{} Now run the command given in the Command column. For \texttt{[agent\_to\_load]} supply ``None'' if this is fold 0 and the name of the current trained agent otherwise.

\paragraph{} There is an optional brief confirmation dialog between you and the robot to confirm its understanding of the objects on the table. It will first touch them each in sequence and ask you to confirm that it touched them left-to-right. Then it will test its ability to recognize touches. One at a time, ask it to recognize a touch and then pick up an object from the user's perspective and ensure the robot reaches out and touches that object afterwards.

\paragraph{} Now the user is ready to be shown in and given instructions. After they have read the instructions, confirm with the robot that you're done testing touches and dialog will begin. By this point, the oracle needs to be ready to transcribe user speech.

\paragraph{} After the two rounds of turns are over, repeat the above with the ``b'' section of this user/fold combination. The user should be shown to another room to wait until the new objects are confirmed with the system and dialog is ready to begin.

\section{Oracle Duties}

\paragraph{} When the user speaks to the robot, you can transcribe what they say using the robot scribe tool by tunneling into the arm robot machine and writing file

\texttt{/home/bwi/catkin\_ws/src/perception\_classifiers/www/communications/[user\_ID].get.in} 

with the user's utterance. Use all lowercase and no punctuation. If the user says a filler word (e.g. ``uh'', ``um'', etc) transcribe the special token ``uh'' which the system knows to ignore.

\paragraph{} When the robot is taking its turn, the user is able to explore objects after hearing the robot describe them. When they're ready to start guessing, they are to move their hands away from the objects and say ``okay''. Transcribe that message to let the robot know it's okay to start recognizing hand movements as guesses.

\end{document}
